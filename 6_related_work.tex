\vspace{-0.1in}  
\section{Related Work}
\vspace{-0.1in}  
Recently abundant works shed light on seamlessly bitrate adaptation. Most of them are mainly based on video on demand(VOD). All the VBR algorithms in VOD can be classified into several categories: rate-based, buffer-based and the combination of both two. Rate-based methods often pick the highest available bitrate lower than the estimated bandwidth \cite{jiang2014improving}, while buffer-based algorithms choose the bitrate according to the buffer level\cite{huang2003adaptive}. 
%If the buffer level is high, it prefers a higher bitrate. However at a low buffer level, a lower bitrate is chosed.
Control theory is also applied to the bitrate adaptation, called MPC. MPC forecasts the future network bandwidth of several slots, and finds the optimal solution during these periods, then applies the first choice \cite{yin2015control}. MPC uses the combination of buffer and throughput. With a large solution space, MPC preforms better than all others.

The difference between VOD and video adaptation in live streaming is the following aspects. First is the time granularity. The time slot lasts for $2$-$10$ seconds in VOD, but in live streaming, the time slot lasts for less than $2$ seconds. The second is the buffer size. In VOD, the buffer usually equals to dozens of seconds, but in live streaming, the buffer is less than $1$ second.

There are also some papers about video adaptation in live streaming scenario. Pires~\etal~\cite{pires2014dash} includes the idea of adaptive streaming in live streaming, but its focus is how to implement in a massive scale and the challenge mainly lies in the resource management. Another paper also researches low-latency live streaming using DASH \cite{bouzakaria2014overhead}. Cicco~\etal\cite{de2011feedback} proposes QAC to switch the encoding parameter using feedback control theory, but the issue lies in the server-client distributing link. All these papers talk little about the video transmission quality.
