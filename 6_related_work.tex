\section{Related Work}
Recently abundant works focus on seamlessly bitrate adaptation of DASH, which is called dynamic adaptive streaming over http. All the VBR algorithm is DASH can be classified into several categories: rate-based and buffer-based and a combination of both two. Rate-based methods often pick the highest available bitrate lower than the estimated bandwidth; buffer-based algorithms choose the bitrate according to the buffer level: at a high buffer level, it prefers to choose a higher bitrate; and at a low buffer level, a lower bitrate instead. Control theory is also applied into the bitrate adaptation, which is called MPC. MPC forecasts the future network bandwidth of several chunks, and finds the optimal solution during these periods, then applies the first choice. MPC preforms better than all others.
There are also some papers about using DASH in live streaming scenario. \cite{pires2014dash} includes the idea of adaptive streaming in live streaming, but its focus is how to implement in a massive scale and the difficulty mainly lies in the resource management.
The difference between DASH and video adaptation in live streaming is the following aspects. One, the time granularity, in DASH, the time slot lasts for $2$-$10$ seconds, but in live streaming, the time slot lasts for less than $2$ seconds; two, the buffer size, DASH is mainly used in VOD, the buffer usually equals to dozens of seconds, but in live streaming, the buffer almost is less than $1$ second. And most important, all of these are at the viewer's side, researches about the broadcaster's side is little. 