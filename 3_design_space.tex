%\section{A closer look at the quality issues}
\subsection{Design Space Insight}
To meet the delay constraints, there are two kinds of solution. One is to limit the timeliness of the sender buffer, i.e., strictly restrain each frame to meet the time requirements; the other may be controlled by scheduling to maintain the average of delay at the target value. Using this method, the largest end-to-end delay may be quite big. In our case, the first one is our choice. We limit the buffer size to $0.9s$.

With buffer size limited, the previous motivating example gives us three intuitions to improve the video streaming quality.

\textbf{Eliminate the dependency between frames.}
By this means, the solution space would be larger and more optimal solutions are expected to be found.

There are two ways to implement this constraint relaxation. A naive approach is to reduce the keyframe interval. For example, if a 2-second interruption starts at the beginning of an 8-second GOP, the whole group are dropped; but if the 8-second GOP is refined to be four 2-second GOP, only one 2-second GOP would be dropped. Thus, the cascading effect would be eliminated. Nevertheless, this approach may be a tradeoff between the minimal frame drop and the video quality, because reducing keyframe interval means less compression in video streaming, to keep a pre-configured bitrate, per-image quality would be degraded (i.e., ``big pixels'').

Another approach is to adapt the GOP selection to the network condition. In details, when the network recovers from an interruption, the first frame transmitted is encoded as I frame, and a new GOP restarts from this first frame. In this way, the new GOP has no dependency on previous (possible dropped) frames, and all its frames are decodable. This approach may need to modify the encoding workflow, which is hard and out of control.

\textbf{Improve the frame drop strategy.} The default strategy in OBS is dropping all P/B frames in buffer when exceeding a threshold. It is something reasonable because if dropping the earlier frames, the following frames cannot be decodable; and if dropping the latest several frames, the earlier frames still exist in buffer. Therefore, the timeliness will be violated. But intuitively, dropping frames within the old GoP, rather than all, may have better performance. It is worth thinking how to design an online frame dropping strategy that approaches the optimal solution. The challenge lies in the complexity of the frame dependency. A brute force solution is impossible due to its time complexity.

\textbf{Adaptive bitrate.} Network failure occurs frequently. Conclusions from figure~\ref{fig:trace-all} validate the fact. Measurements show that commercial applications only use constant bitrate(CBR) or ABR, which means the actual bitrate varies among the target value, at most $20\%$ lower or higher of the target bitrate. These two methods cannot follow the dynamic bandwidth, which would bring tremendous frame dropping when bandwidth falls down, especially in the case where the bandwidth drop lasts for a certain while. One possible solution is similar to DASH in VOD scenario, applying adaptive bitrate in broadcaster's side. In our case, the bitrate differs between two GoPs, which means we would decide a bitrate for each GoP. Introducing bitrate adaptation maybe dramatically cut down the frame dropping.

\iffalse
\begin{itemize}
\item If we can relax the dependency between frames, the solution space would be larger and more optimal solutions are expected to be found. That is, we can relax the decodability constraints to be $d_i = 1, \forall i$.

\item We can relax queue length constraint, i.e., making $T_1$ larger. This change similarly increases the space of possible solutions.

\item The frame drop strategy can be improved. That is, compared with the naive strategy in OBS (dropping all P/B when exceeding a threshold), selectively choosing frames to drop in IP would give a more optimal solution.
\end{itemize}
\fi

