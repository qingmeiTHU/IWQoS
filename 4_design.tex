\newcommand{\Mod}[1]{\text{ (mod } #1\text{)}}
\vspace{-0.18in}
\section{\name Solution}
\vspace{-0.05in}
Based on the intuitions above, we review the design of RTMP and customize the protocol in three levels. First, we tune the GOP size to reduce the dependency between frame so that ``cascading effect'' is mitigated; second, we improve the frame drop strategy within GOP to avoid the drop as much as possible. These two methods are targeted at short throughput drop. Third, we devise a bitrate adaptation algorithm across GOP for long-term wireless bandwidth degradation.

%In particular, GVBR optimizes the adaptation at frame-level (Section~\ref{subsec:drop-strategy}), as well as GOP-level
\subsection{Determining GOP Size}
Large GOP size causes ``cascading effect'' when the network suffers from transient bandwidth drop; but small GOP causes high compression ratio, causing frame resolution to be small. Thus, we find the best GOP size as the tradeoff between drop resistance and resolution.

The influence of GOP size on frame drops is measured by controlled experiments. We replay the stored video with controlled network conditions, tune GOP size in each replay, and figure out the GOP size with the least frame drops.

As for the video resolution, SSIM(structural similarity) is used as the metric. It is a method for predicting quality of video, and is used for measuring
the similarity of two images~\cite{wang2004image}. Then we vary GOP size, and figure out the minimum GOP size that can keep SSIM to be within [95\%-100\%] of the SSIM of the original video. This computation can be done offline in a cloud, CDN server, or in the streaming device.

As for the video resolution, SSIM(structural similarity) is used as the metric. It is a method for predicting quality of video, and is used for measuring
the similarity of two images~\cite{wang2004image}. Then we vary GOP size, and figure out the minimum GOP size that can keep SSIM to be within [95\%-100\%] of the SSIM of the original video. This computation can be done offline in a cloud, CDN server, or in the streaming device.

\vspace{-0.05in}
\subsection{Smart Drop Strategy}
\label{subsec:drop-strategy}
In the runtime with GOP size configure, a frame drop logic is needed for transient network degradation. We first theoretically figure out the best possible drop decision (i.e., best video quality) that can be made assuming the network degradation is known beforehand. Then we design an online algorithm that has low complexity and is suitable for mobile devices.

\subsubsection{Problem Analysis}
\begin{table}[tb]
\footnotesize
\centering
\caption{Terminology in Integer Program}
\label{tbl:term}
{\setlength{\tabcolsep}{1pt}
\begin{tabular}{|c|c|l|}
\hline
\textbf{Symbol} & \textbf{Type} & \textbf{Meaning}                      \\ \hline
$i$               & index         & frame index                           \\ \hline
$j$               & index         & time index                            \\ \hline
$x_{ij}$             & variable      & whether frame $i$ is in queue at time $j$ \\ \hline
$y_{ij}$             & variable      & whether frame $i$ is sent at time $j$     \\ \hline
$z_{ij}$             & variable      & whether frame $i$ is dropped at time $j$  \\ \hline
$T$               & const         & decision time                        \\ \hline
$T_1$             & const       & max time when a frame keeps ``fresh'' \\ \hline
$C_j$              & const         & network bandwidth at time $j$           \\ \hline
$N$               & const         & key frame interval                      \\ \hline
$S$            & const         & each frame size                        \\ \hline
$M_{j}$       & const         & frame index that can be send at time $j$ \\ \hline
$R_{i}$        & const         & bitrate of the $i$ frame               \\ \hline
\end{tabular}}
\end{table}

\begin{figure}[tb]
\centering
{\setlength{\tabcolsep}{3pt}
\begin{tabular}{|l|l|}
\hline
\multicolumn{2}{|c|}{ maximize $\Sigma_i y_{iT}$, subject to} \\ \hline \hline
 $x_{ij}+y_{ij}+z_{ij} = 0, \forall j<i$       & (1)        \\
 $x_{ij}+y_{ij}+z_{ij} = 1, \forall j\geq i$   & (2)       \\
 $x_{ij} \geq x_{i,j+1}, \forall j\geq i$      & (3)        \\
 $y_{ij} \leq y_{i,j+1}, \forall j\geq i$      & (4)        \\
 $z_{ij} \leq z_{i,j+1}, \forall j\geq i$      & (5)        \\ \hline
 $y_{ij} = \max\{1,{1-z_{i,j-1}}\}, \forall j, i \leq M_{j}$  & (6)   \\ \hline
 $y_{ij}+z_{ij} = 1 ,\forall j>i+T_2$          & (7)        \\ \hline
 $y_{i+1,T} \geq y_{iT}, \forall i \not\equiv N-1 (\text{mod}N)$ &(8) \\ \hline
\end{tabular}}
\caption{Frame Drop Strategy}
\label{fig:ip-program} \mylabel{fig:ip-program}
\end{figure}

We first figure out the best video quality that can be achieved by making frame drop decisions.
Assuming the pace of video frame and network bandwidth are known, there exists an optimal scheduling regarding maximizing audience QoE within the system constraints (bandwidth and queue timeliness length). A group of pictures comprise three kinds of frames, namely I/P/B frames; for simplicity here, we ignore the B frame to study the fundamental problem. The problem can be formulated by integer programming (Figure~\ref{fig:ip-program}). Symbols are defined in Table~\ref{tbl:term}. We discretize time into time stamps from $0$ to $T$, and assume the frame with index $i$ is generated at time $i$. We define $x_{ij}$, $y_{ij}$, $z_{ij}$ as 0/1 variables to describe whether a packet is in the queue, sent or dropped respectively.

\textbf{Frame conservation constraints.}
Frame $i$ is generated at time $i$, and after that, it is either in the queue or sent or dropped (1-2).
After a packet is removed from the queue, it would never be enqueued (3).
After a packet is sent/dropped, it is permanently sent/ dropped afterward (4-5).

\textbf{Bandwidth constraints.}
The sending strategy $y_{ij}$, is also an interesting and important problem in improving frame drops. However, in this paper we assume that the broadcaster sends as many as possible when meeting the network capacity constraints. In addition, the frame that can be send must be in the buffer. In line with the constraints, the max frame index $M_{j}$, is calculated by maximizing the function.
\vspace{-0.1in}
\begin{align}
M_j = argmax \Sigma_k (1-y_{k,j-1})(1-z_{k,j-1}) \leq C_{j}
\end{align}

\textbf{Timeliness constraint.}
A frame is ``fresh'' if it is sent with in ``$T_1$''. That is, a frame is either sent or drop after time $T_1$ of its generation (7).

\textbf{Decodability constraints.} The final delivered frames must be decodable. Otherwise, they would be a waste of network bandwidth. I frames are always decodable. A P frame is decodable if and only if its preceding I or P frame is decodable (8).

\textbf{Optimization goal.} The goal of the IP model is to maximize the delivered frames.
Compared with prior work~\cite{singh2004dynamic}, this IP model has timeliness and decodability in consideration, thus it is more suitable for personalized live streaming.


\subsubsection{Greedy Algorithm}
\begin{algorithm}[tb]
\caption{GreedyDrop Algorithm}
\label{alg:greedy-drop}\mylabel{alg:greedy-drop}
\begin{algorithmic}[1]
\State \textbf{Input:} {frame, bandwidth}
\State T1 := 0.9s
\If{frame is I frame}
\State dropPFrame := False
\State \Call{Enqueue}{queue, frame}
\State timespan:= timespan + 1
\EndIf
\If{frame is P frame}
\If{dropPFrame or timespan $>$ T1}
\If{I frame not exist in buffer}
\State dropPFrame := True
\EndIf
\State \Call{Drop}{frame}, drop all the P frames until the next I frame
\Else
\State \Call{Enqueue}{queue, frame}
\State timespan := timespan + 1
\EndIf
\EndIf
\State timespan := timespan - \Call{Time-Send}{bandwidth}
\end{algorithmic}
\end{algorithm} 
IP can achieve the offline optimal. Nevertheless long-term bandwidth is unknowable ahead of time, and the computational complexity is too high for mobile devices to run it. Thus, IP cannot be applied in practice. Consequently, an online drop strategy is necessary.

Considering the situation where two or more GOPs coexist in buffer, we propose a modified dropping algorithm, GreedyDrop (Algorithm~\ref{alg:greedy-drop}). Differing from dropping all the P frames in buffer by default, GreedyDrop drops all the P frames until the next keyframe. Hence the latest GOP can be reserved and our algorithm avoid frame dropping at least one GOP.

In experiments where we control network bandwidth, we can compare the effect of the IP approach and the greedy algorithm.

\subsection{Adaptive Bitrate}
\label{subsec:adaptive-bitrate}
\subsubsection{Problem Formulation}
\begin{table}[tb]
\centering
\caption{Terminology in Adaptive Bitrate}
\label{tbl:vbrval}
{\setlength{\tabcolsep}{1pt}
\begin{tabular}{|c|c|l|}
\hline
\textbf{Symbol} & \textbf{Type} & \textbf{Meaning}                      \\ \hline
$j$               & index         & frame index                            \\ \hline
$R_j$             & variable      & the bitrate of frame $j$ \\ \hline
$N_j$             & variable      & No. of the GoPs at time $j$     \\ \hline
$D_j$             & variable      & whether frame $i$ is dropped at time $j$  \\ \hline
$Send_j$             & variable   & No. of frame send at time $j$ \\ \hline
$C_j$              & variable         & network bandwidth at time $j$           \\ \hline
$T_k^j$               & variable         &the remaining time of $k$ GoP at time $j$  \\ \hline
$R_k^j$            & variable         & the bitrate of $k$ GoP at time $j$  \\ \hline
$Drop_j$       & variable & whether the drop would happen at time $j$ \\ \hline
$T$               & const         & decision time                        \\ \hline\end{tabular}}
\end{table}

\begin{eqnarray}
&Max& \sum R_j- \alpha\sum|R_{j+1}-R_j|-\beta\sum D_j
\label{vbr-formulation} \mylabel{vbr-formulation}
\end{eqnarray}
\\ subject to
\begin{eqnarray}
% \nonumber % Remove numbering (before each equation)
&& R_{j+1}=R_j, \forall mod(j,M)\not\equiv M-1 \label{vbr-bitrate}\\
&& S_j = argmax{\sum_k R_k^j*T_k^j \leq C_j}, \forall j \label{vbr-send} \\
&& Rest_j = (C_j- \sum_{S_j} R_k^j*T_k^j)/R_{S_j+1}^j, \forall j \\
&& F_j = sgn(\sum_{S_j+1} T_k^j - Rest_j-T_1), \forall j \label{vbr-drop}\\
&& D_j = F_j*(T_{S_j+1}^j-Rest_j), \forall j \label{vbr-drop-no} \\
&& N_{j+1}=N_j-S_j-F_j+1-sgn(mod(j,M)), \forall j \label{vbr-gop-no}\\
&& R_k^{j+1}=R_{k+S_j+F_j}^j, \forall j, k\in \{1,N_j-S_j-F_j\} \label{vbr-bitrate-next}\\
&& R_{N_j-S_j-F_j+1}^{j+1} = R_{j+1}, \forall mod(j,M) \equiv 0 \label{vbr-bitrate-spec} \\
&& T_k^{j+1} = T_{k+S_j+F_j}^j, \forall j, k\in\{1, N_j-S_j-F_j\} \label{vbr-time-next} \\
&& T_{N_j-S_j-F_j}^{j+1} = T_{N_j-S_j-F_j}^{j+1} - D_j - Rest_j , \forall j \label{vbr-time-spec} \\
&& T_{N_j-S_j-F_j+1}^{j+1}=1, \forall mod(j,M)\equiv 0 \label{vbr-time-spec2} \\
\end{eqnarray}


To further deal with long-term bandwidth fading, we adopt an adaptive bitrate approach. We still first explore the theoretically best quality that can be achieved by tuning the bitrate. We then design an online algorithm for mobile devices.

We introduce a variable $R_{i}$. $R_{i}$ represents the bitrate of the $i$ frame. On the basis of adopting GreedyDrop as our drop strategy, the utility function can be formulated as in Figure~\ref{vbr-formulation}.
% equation \ref{vbr-formulation},
The first item is the bitrate utility, the second represents the bitrate switch penalty, the last one equals the frame drops penalty. Variables are all defined in Table~\ref{tbl:vbrval}. Variables $\alpha$ and $\beta$ are the utility parameter of bitrate switch and frame drops. $sgn$ is the sign function. When the variable greater than zero, it equals 1; otherwise it equals zero. $mod$ is the operation of taking the remainder.

\textbf{Bitrate Constraint.} Constraint (1) requires that bitrate within one GOP should be identical.

\textbf{Bandwidth Constraint.} Equation (2) calculates the maximum number of sendable GOPs within the limited bandwidth.

\textbf{Timeliness Constraint.} Constraint (3) judges whether the remaining time after sending exceeds the buffer threshold and constraint (4) give the number of dropped frames in time $j$.

\textbf{State Transition.} Constraints (7-11) reflect the state transition of the bitrate and remaining time of several GOPs in the buffer. Equations (6) describes the number of GOPs in the next time slot $j+1$, and the last two items $1-mod(j,M)$ represents whether the $j-th$ frame is the keyframe.

Offline optimal solution is hard to calculate. Assume for each GOP, the broadcaster can choose one from total $M$ bitrate candidates. For a $T$ GOP decision, the computation complexity equals $M^T$, an exponential complexity.

\subsubsection{Effective Solution}
\begin{algorithm}[h]
\caption{Greedy Video Adaptation Workflow}
\label{alg:greedy-vbr}\mylabel{alg:greedy-vbr}
\begin{algorithmic}[1]
\State Initialize
\For{j=1 to T}
\State $C_j := ThroughpurPred(C_{[\tau,j-1]})$
\State $R_j := FindClosestBr(C_j)$
\State $D_j := f(C_j, R_j)$
\EndFor
\end{algorithmic}
\end{algorithm} 
\textbf{Algorithm Description.} Issue with exponential complexity is hard to calculate in limited time. In addition, the offline optimal is on the basis of off-the-shelf knowledge of future bandwidth. Such long-term bandwidth prediction is inaccurate. An intuitive idea is to change the bitrate following the bandwidth. Moreover, the remaining data size in the buffer can also be adopted. At time $j$, the broadcaster carries out the following two key steps, as shown in Greedy Variable Bitrate (GVBR) (Algorithm ~\ref{alg:greedy-vbr}). $\eta$ is the tuning parameter of frame drops and bitrate.

$1.$ Bandwidth estimation. According to Festive~\cite{jiang2014improving} and MPC~\cite{yin2015control}, harmonic mean is a useful method of estimating the future bandwidth. With more accurate bandwidth estimations, our method performs much more satisfactory.

$2.$ Bitrate selection. To avoid frequent frame dropping, an appropriate bitrate is essential. Given the future bandwidth $C_j$ and the data size in buffer $Rest$, a heuristic method is to choose the largest available bitrate lower than $(C_j-Rest)/\eta$.

GVBR use a suite of solutions that improve broadcaster-side QoE, both in GOP level and within GOP. 